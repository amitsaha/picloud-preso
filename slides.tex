\documentclass{beamer} \usepackage{fancyvrb} \usepackage{listings}
\usepackage{color} 
\usepackage{caption}
\usepackage{wrapfig}

\mode<presentation>  
    { 

      \usetheme{CambridgeUS}}
    \usepackage[english]{babel} \usepackage[latin1]{inputenc}

    \usepackage{times} \usepackage[T1]{fontenc}

    \title{PiCloud: Cloud Computing Simplified}

    \subtitle{A hands-on quick overview}

    \author{Amit Saha \\ PyCon Australia'12 \\ @echorand \\ }
    

    %\date {October 1, 2009/ LFY}

    \begin{document}

    \begin{frame}
      \titlepage
    \end{frame}

    \begin{frame}{About Me}
      \begin{itemize}
      \item PiCloud (and Python) enthusiast
      \item Freelance Technical Writer
      \item Fedora project contributor: Scientific Spin, Google Summer
        of Code, etc.
      \item Blog: \url{http://echorand.me}
      \end{itemize}
    \end{frame}

    \begin{frame}{Outline}
      \tableofcontents
      \end{frame}

    \begin{frame}{Slides and demos}
      \begin{center}
        \textcolor{blue}{Get it:} git clone https://github.com/amitsaha/picloud-preso
      \end{center}
    \end{frame}


    \section{Introduction}

    \begin{frame}{What is PiCloud?}
      \begin{itemize}
      \item Cloud Computing solution
      \item Primarily acessible via Python library:
        \textcolor{blue}{import cloud}
      \item Commercial offering, but has 20 free core hours/month for
        all
      \end{itemize}
    \end{frame}

    \begin{frame}{Key Features}
      \begin{itemize}
      \item \textbf{Automated Deployment:} Your code (along with
        modules) magically transported \pause
      \item \textbf{Choice of computing power}: Cores of different
        capabilities
      \item \textbf{Scientific Computing ready}: SciPy and NumPy
        (others can be installed)
      \item REST APIs, Environments, Cron Jobs, S3 Storage ..
      \end{itemize}
    \end{frame}

    \section{Setup and Use}

    \begin{frame}{Easy Setup}
      \begin{itemize}
      \item Register at
        \url{https://www.picloud.com/accounts/register/}
      \item \textcolor{blue}{\$sudo pip-python install cloud}
      \item \textcolor{blue}{\$ picloud setup} (Your email address)
      \item Gives you an API key (keep it safe)
      \end{itemize}

    \end{frame}

    \begin{frame}{Sanity check}
      \centering(IPython notebook available in the slide repository)
      \lstset{
  basicstyle=\footnotesize\ttfamily, % Standardschrift
  %numbers=left,               % Ort der Zeilennummern
  numberstyle=\tiny,          % Stil der Zeilennummern
  %stepnumber=2,               % Abstand zwischen den Zeilennummern
  numbersep=5pt,              % Abstand der Nummern zum Text
  tabsize=2,                  % Groesse von Tabs
  extendedchars=true,         %
  breaklines=true,            % Zeilen werden Umgebrochen
  keywordstyle=\color{blue},
  frame=single,         
  %        keywordstyle=[1]\textbf,    % Stil der Keywords
  %        keywordstyle=[2]\textbf,    %
  %        keywordstyle=[3]\textbf,    %
  %        keywordstyle=[4]\textbf,   \sqrt{\sqrt{}} %
  stringstyle=\color{black}\ttfamily, % Farbe der String
  showspaces=false,           % Leerzeichen anzeigen ?
  showtabs=false,             % Tabs anzeigen ?
  xleftmargin=17pt,
  framexleftmargin=17pt,
  framexrightmargin=5pt,
  framexbottommargin=4pt,
  %backgroundcolor=\color{lightgray},
  showstringspaces=false      % Leerzeichen in Strings anzeigen ?        
}
\lstloadlanguages{% Check Dokumentation for further languages ...
  %[Visual]Basic
  %Pascal
  %C
  %C++
  %XML
  %HTML
  %Java
  Python
}
%\DeclareCaptionFont{blue}{\color{blue}} 
%\captionsetup[lstlisting]{singlelinecheck=false}%, labelfont={blue}, textfont={blue}}

      \lstinputlisting[language=Python]{code/picloud_sanity.py}
    \end{frame}

    \begin{frame}{Sanity Check: Demo}
      \begin{center}
        \textcolor{blue}{DEMO}
      \end{center}
    \end{frame}
    

    %\tiny \centering(IPython notebook available in the slide repository)
    \begin{frame}{Sorting in the Cloud}
      \lstset{
  basicstyle=\footnotesize\ttfamily, % Standardschrift
  %numbers=left,               % Ort der Zeilennummern
  numberstyle=\tiny,          % Stil der Zeilennummern
  %stepnumber=2,               % Abstand zwischen den Zeilennummern
  numbersep=5pt,              % Abstand der Nummern zum Text
  tabsize=2,                  % Groesse von Tabs
  extendedchars=true,         %
  breaklines=true,            % Zeilen werden Umgebrochen
  keywordstyle=\color{blue},
  frame=single,         
  %        keywordstyle=[1]\textbf,    % Stil der Keywords
  %        keywordstyle=[2]\textbf,    %
  %        keywordstyle=[3]\textbf,    %
  %        keywordstyle=[4]\textbf,   \sqrt{\sqrt{}} %
  stringstyle=\color{black}\ttfamily, % Farbe der String
  showspaces=false,           % Leerzeichen anzeigen ?
  showtabs=false,             % Tabs anzeigen ?
  xleftmargin=17pt,
  framexleftmargin=17pt,
  framexrightmargin=5pt,
  framexbottommargin=4pt,
  %backgroundcolor=\color{lightgray},
  showstringspaces=false      % Leerzeichen in Strings anzeigen ?        
}
\lstloadlanguages{% Check Dokumentation for further languages ...
  %[Visual]Basic
  %Pascal
  %C
  %C++
  %XML
  %HTML
  %Java
  Python
}
%\DeclareCaptionFont{blue}{\color{blue}} 
%\captionsetup[lstlisting]{singlelinecheck=false}%, labelfont={blue}, textfont={blue}}

      \lstinputlisting[language=Python]{code/picloud_sort.py}
    \end{frame}

    \begin{frame}{A PiCloud decorator}
      \lstset{
  basicstyle=\footnotesize\ttfamily, % Standardschrift
  %numbers=left,               % Ort der Zeilennummern
  numberstyle=\tiny,          % Stil der Zeilennummern
  %stepnumber=2,               % Abstand zwischen den Zeilennummern
  numbersep=5pt,              % Abstand der Nummern zum Text
  tabsize=2,                  % Groesse von Tabs
  extendedchars=true,         %
  breaklines=true,            % Zeilen werden Umgebrochen
  keywordstyle=\color{blue},
  frame=single,         
  %        keywordstyle=[1]\textbf,    % Stil der Keywords
  %        keywordstyle=[2]\textbf,    %
  %        keywordstyle=[3]\textbf,    %
  %        keywordstyle=[4]\textbf,   \sqrt{\sqrt{}} %
  stringstyle=\color{black}\ttfamily, % Farbe der String
  showspaces=false,           % Leerzeichen anzeigen ?
  showtabs=false,             % Tabs anzeigen ?
  xleftmargin=17pt,
  framexleftmargin=17pt,
  framexrightmargin=5pt,
  framexbottommargin=4pt,
  %backgroundcolor=\color{lightgray},
  showstringspaces=false      % Leerzeichen in Strings anzeigen ?        
}
\lstloadlanguages{% Check Dokumentation for further languages ...
  %[Visual]Basic
  %Pascal
  %C
  %C++
  %XML
  %HTML
  %Java
  Python
}
%\DeclareCaptionFont{blue}{\color{blue}} 
%\captionsetup[lstlisting]{singlelinecheck=false}%, labelfont={blue}, textfont={blue}}

      \lstinputlisting[language=Python]{code/decorator.py}
    \end{frame}

    \begin{frame}{Using the decorator}
      \lstset{
  basicstyle=\footnotesize\ttfamily, % Standardschrift
  %numbers=left,               % Ort der Zeilennummern
  numberstyle=\tiny,          % Stil der Zeilennummern
  %stepnumber=2,               % Abstand zwischen den Zeilennummern
  numbersep=5pt,              % Abstand der Nummern zum Text
  tabsize=2,                  % Groesse von Tabs
  extendedchars=true,         %
  breaklines=true,            % Zeilen werden Umgebrochen
  keywordstyle=\color{blue},
  frame=single,         
  %        keywordstyle=[1]\textbf,    % Stil der Keywords
  %        keywordstyle=[2]\textbf,    %
  %        keywordstyle=[3]\textbf,    %
  %        keywordstyle=[4]\textbf,   \sqrt{\sqrt{}} %
  stringstyle=\color{black}\ttfamily, % Farbe der String
  showspaces=false,           % Leerzeichen anzeigen ?
  showtabs=false,             % Tabs anzeigen ?
  xleftmargin=17pt,
  framexleftmargin=17pt,
  framexrightmargin=5pt,
  framexbottommargin=4pt,
  %backgroundcolor=\color{lightgray},
  showstringspaces=false      % Leerzeichen in Strings anzeigen ?        
}
\lstloadlanguages{% Check Dokumentation for further languages ...
  %[Visual]Basic
  %Pascal
  %C
  %C++
  %XML
  %HTML
  %Java
  Python
}
%\DeclareCaptionFont{blue}{\color{blue}} 
%\captionsetup[lstlisting]{singlelinecheck=false}%, labelfont={blue}, textfont={blue}}

      \lstinputlisting[language=Python]{code/picloud_decorator_fun.py}
    \end{frame}

    \begin{frame}{A messy graphic}
        \begin{center}
          \includegraphics[scale=0.4, keepaspectratio=true]{job_complex.PNG}
        \end{center}
        \textbf{Source:} \url{http://docs.picloud.com/tech_overview.html\#take-away}
      \end{frame}

    \section{PiCloud Features}

    \begin{frame}{Moving persistent data}
      \begin{itemize}
      \item \textcolor{blue}{cloud.files} module
      \item Store a file: \textcolor{blue}{cloud.files.put()}
      \item Retrieve a file: \textcolor{blue}{cloud.files.get()}
      \item List all files: \textcolor{blue}{cloud.files.list()}
      \end{itemize}
    \end{frame}

    \begin{frame}{Persistent data demo}
      \lstset{
  basicstyle=\footnotesize\ttfamily, % Standardschrift
  %numbers=left,               % Ort der Zeilennummern
  numberstyle=\tiny,          % Stil der Zeilennummern
  %stepnumber=2,               % Abstand zwischen den Zeilennummern
  numbersep=5pt,              % Abstand der Nummern zum Text
  tabsize=2,                  % Groesse von Tabs
  extendedchars=true,         %
  breaklines=true,            % Zeilen werden Umgebrochen
  keywordstyle=\color{blue},
  frame=single,         
  %        keywordstyle=[1]\textbf,    % Stil der Keywords
  %        keywordstyle=[2]\textbf,    %
  %        keywordstyle=[3]\textbf,    %
  %        keywordstyle=[4]\textbf,   \sqrt{\sqrt{}} %
  stringstyle=\color{black}\ttfamily, % Farbe der String
  showspaces=false,           % Leerzeichen anzeigen ?
  showtabs=false,             % Tabs anzeigen ?
  xleftmargin=17pt,
  framexleftmargin=17pt,
  framexrightmargin=5pt,
  framexbottommargin=4pt,
  %backgroundcolor=\color{lightgray},
  showstringspaces=false      % Leerzeichen in Strings anzeigen ?        
}
\lstloadlanguages{% Check Dokumentation for further languages ...
  %[Visual]Basic
  %Pascal
  %C
  %C++
  %XML
  %HTML
  %Java
  Python
}
%\DeclareCaptionFont{blue}{\color{blue}} 
%\captionsetup[lstlisting]{singlelinecheck=false}%, labelfont={blue}, textfont={blue}}
 \lstinputlisting[language=Python]{code/picloud_filedemo.py}
    \end{frame}

    \begin{frame}{Evolutionary Algorithms in the Cloud}
      \begin{itemize}
      \item \textcolor{red}{Pyevolve}: Python Evolutionary Algorithm library
      \item More than one ways to parallelize
      \item Execute the algorithm on PiCloud infrastructure
      \item For more information:
        \url{http://pyevolve.sourceforge.net/}
        
      \end{itemize}
    \end{frame}
    
    \begin{frame}{Pyevolve + PiCloud}
      \lstset{
  basicstyle=\footnotesize\ttfamily, % Standardschrift
  %numbers=left,               % Ort der Zeilennummern
  numberstyle=\tiny,          % Stil der Zeilennummern
  %stepnumber=2,               % Abstand zwischen den Zeilennummern
  numbersep=5pt,              % Abstand der Nummern zum Text
  tabsize=2,                  % Groesse von Tabs
  extendedchars=true,         %
  breaklines=true,            % Zeilen werden Umgebrochen
  keywordstyle=\color{blue},
  frame=single,         
  %        keywordstyle=[1]\textbf,    % Stil der Keywords
  %        keywordstyle=[2]\textbf,    %
  %        keywordstyle=[3]\textbf,    %
  %        keywordstyle=[4]\textbf,   \sqrt{\sqrt{}} %
  stringstyle=\color{black}\ttfamily, % Farbe der String
  showspaces=false,           % Leerzeichen anzeigen ?
  showtabs=false,             % Tabs anzeigen ?
  xleftmargin=17pt,
  framexleftmargin=17pt,
  framexrightmargin=5pt,
  framexbottommargin=4pt,
  %backgroundcolor=\color{lightgray},
  showstringspaces=false      % Leerzeichen in Strings anzeigen ?        
}
\lstloadlanguages{% Check Dokumentation for further languages ...
  %[Visual]Basic
  %Pascal
  %C
  %C++
  %XML
  %HTML
  %Java
  Python
}
%\DeclareCaptionFont{blue}{\color{blue}} 
%\captionsetup[lstlisting]{singlelinecheck=false}%, labelfont={blue}, textfont={blue}}
 \lstinputlisting[language=Python]{code/ga_driver.py}
    \end{frame}

    \begin{frame}{Automatic Deployment + Moving files}
      \begin{center}
        \textcolor{blue}{DEMO}
      \end{center}
    \end{frame}



    \begin{frame}{REST API}
      \begin{itemize}
      \item Publish your functions via a \textcolor{red}{REST API}
      \item Language independent access to your Python functions
      \item Most of the cloud library functions have REST analogs
      \end{itemize}
    \end{frame}

    \begin{frame}{REST API: Publishing a function}
      \lstset{
  basicstyle=\footnotesize\ttfamily, % Standardschrift
  %numbers=left,               % Ort der Zeilennummern
  numberstyle=\tiny,          % Stil der Zeilennummern
  %stepnumber=2,               % Abstand zwischen den Zeilennummern
  numbersep=5pt,              % Abstand der Nummern zum Text
  tabsize=2,                  % Groesse von Tabs
  extendedchars=true,         %
  breaklines=true,            % Zeilen werden Umgebrochen
  keywordstyle=\color{blue},
  frame=single,         
  %        keywordstyle=[1]\textbf,    % Stil der Keywords
  %        keywordstyle=[2]\textbf,    %
  %        keywordstyle=[3]\textbf,    %
  %        keywordstyle=[4]\textbf,   \sqrt{\sqrt{}} %
  stringstyle=\color{black}\ttfamily, % Farbe der String
  showspaces=false,           % Leerzeichen anzeigen ?
  showtabs=false,             % Tabs anzeigen ?
  xleftmargin=17pt,
  framexleftmargin=17pt,
  framexrightmargin=5pt,
  framexbottommargin=4pt,
  %backgroundcolor=\color{lightgray},
  showstringspaces=false      % Leerzeichen in Strings anzeigen ?        
}
\lstloadlanguages{% Check Dokumentation for further languages ...
  %[Visual]Basic
  %Pascal
  %C
  %C++
  %XML
  %HTML
  %Java
  Python
}
%\DeclareCaptionFont{blue}{\color{blue}} 
%\captionsetup[lstlisting]{singlelinecheck=false}%, labelfont={blue}, textfont={blue}}

      \lstinputlisting[language=Python]{code/picloud_square_rest.py}
    \end{frame}

    \begin{frame}{REST API: Invoking the published functions}
      \begin{itemize}
      \item Using the end-point, first get the job ID
      \item Use this Job ID to get the result
      \item Appropriate requests can be made using \textcolor{red}{curl} or any other
        client capable of invoking REST APIs
      \end{itemize}
    \end{frame}

    \begin{frame}{Environments}
      \begin{itemize}
      \item Non-Python software packages
      \item Python libraries with native-dependencies (not already
        installed)
      \item Your personal sandbox in PiCloud (based on Ubuntu Linux)
      \item Specify environment to execute your code in
      \end{itemize}
    \end{frame}


    \begin{frame}{Job Monitoring and Management}
      \begin{itemize}
      \item Get information about a job:
        \textcolor{blue}{cloud.info()}
      \item Kill, Delete jobs {\textcolor{blue}{cloud.kill(),
          cloud.delete()}}
      \item Web interface has functions for managing your jobs,
        crons, analytics, payment ..
      \item Manage your account using
        \textcolor{blue}{cloud.account} module
      \end{itemize}

    \end{frame}


    \section{Ending notes}

    \begin{frame}{Summarize}
      \begin{itemize}
      \item A truly simple way to harness and play around with cloud
        computing
      \item Code is there to explore
      \item Simulator to help you test your code
      \item Just go ahead and \textcolor{blue}{import cloud}!
      \end{itemize}

    \end{frame}
    
    \begin{frame}{Resources}
      \begin{itemize}
      \item PiCloud Homepage: \url{http://www.picloud.com}
      \item Technical Overview: \url{http://docs.picloud.com/tech_overview.html}
      \item PiCloud Pitfalls:
        \url{http://docs.picloud.com/client_pitfall.html}
      \item PiCloud Documentation: \url{http://docs.picloud.com/}
      \item PiCloud FAQ: \url{http://www.picloud.com/faq/}
      \item PiCloud: An Easy Way to the Cloud:
        \url{http://bit.ly/GZDxZB}
      \item Presentation and Code:
        \url{https://github.com/amitsaha/picloud-preso}
      \end{itemize}
    \end{frame}

    \end{document}


